\chapter{Практика}
Найдем положения равновесия нашей системы:

$$ \left\{
\begin{aligned}
   \dot{u} = u \left(1 - u\right) - \frac{\alpha  u  v}{u+v},\\
   \dot{v} = \beta  v \left(1 - \gamma  \frac{v}{u + \delta}\right).\\
\end{aligned}
\right. $$

В итоге мы получаем 4 пары:

$\left ( 0, \frac{\delta }{\gamma} \right )$ ,
$\left ( 1, 0 \right )$ 

А также пары с большими значениями:

\[ \begin{aligned}
\Bigl ( &\frac{\sqrt{\alpha ^2-2 \alpha  (2 \gamma \delta +\gamma +\delta +1)+(\gamma +\delta +1)^2}+\alpha -\gamma  +\delta  -1}{2 (\gamma  +1)},\\
&\frac{-\sqrt{\alpha ^2-2 \alpha  (2  \gamma  \delta +\gamma +\delta +1)+(\gamma +\delta +1)^2}-\alpha +2 \gamma  \delta +\gamma +\delta +1}{2 \gamma  (\gamma +1)} \Bigr ) 
\end{aligned} \]

и

\[ \begin{aligned}
\Bigl ( &\frac{\sqrt{\alpha ^2-2 \alpha  (2 \gamma  \delta +\gamma +\delta +1)+(\gamma +\delta +1)^2}-\alpha +\gamma -\delta +1}{2 (\gamma +1)},\\ &\frac{\sqrt{\alpha ^2-2 \alpha  (2 \gamma  \delta +\gamma +\delta +1)+(\gamma +\delta +1)^2}-\alpha +2 \gamma \delta +\gamma +\delta +1}{2 \gamma  (\gamma +1)} \Bigr ) 
\end{aligned} \]

Для простоты решения мы не будем работать с большими числами, а выберем пару $\left ( 0, \frac{\delta }{\gamma} \right )$. Мы выбираем ее, так как в ней присутствует два параметра системы, в отличие от пары $\left ( 1, 0 \right )$. 

Дальше нам необходимо сделать сдвиг системы на выбранное положение равновесия. Где у нас $u$ заменялся на $0 + u_1$, а $v$ на $\frac{\delta }{\gamma} + v_1$. Мы получим систему:

$ \left\{
\begin{aligned}
	&\dot{u_1} = \frac{\alpha  \gamma  u_1^2}{\delta +\gamma  (u_1+v_1)}-u_1 (\alpha +u_1-1),\\
	&\dot{v_1} = \frac{\beta  (u_1-\gamma  v_1) (\delta +\gamma  v_1)}{\gamma (\delta +u_1)}.\\
\end{aligned}
\right. $

Линеаризовав нашу систему мы поличили матрицу:

$\left(
\begin{array}{cc}
 1-\alpha  & 0 \\
 \frac{\beta }{\gamma } & -\beta  \\
\end{array}
\right)$

Найдем собвстенные значения матрицы и получим: 
$\lambda_1 = 1-\alpha $ и $\lambda_2 = -\beta$ 

Далее к каждому значению $\lambda$ найдем собственный вектор $\vec{v}$

$\vec{v_1} = \left(
\begin{array}{c}
 \gamma  \\
 1 \\
\end{array}
\right)$, а 
$\vec{v_2} = \left(
\begin{array}{c}
 0  \\
 1 \\
\end{array}
\right)$
%%% Local Variables: 
%%% mode: latex
%%% TeX-PDF-mode: t
%%% TeX-master: "diploma"
%%% End: 
