\chapter{Практика}
Найдем положения равновесия нашей системы:

$$ \left\{
\begin{aligned}
   u \left((\frac{\alpha  v}{u+v}+u-1\right)=0,\\
   \beta  v \left(\frac{\gamma  v}{\delta +u}-1\right)=0.\\
\end{aligned}
\right. $$

В итоге мы получаем 4 пары:

$\left ( 0, \frac{\delta }{\gamma} \right )$ ,
$\left ( 1, 0 \right )$ 

А также пары с большими значениями:

$\left ( \frac{\sqrt{\alpha ^2-2 \alpha  (2 \gamma \delta +\gamma +\delta +1)+(\gamma +\delta +1)^2}+\alpha -\gamma  +\delta  -1}{2 (\gamma  +1)}, \frac{-\sqrt{\alpha ^2-2 \alpha  (2  \gamma  \delta +\gamma +\delta +1)+(\gamma +\delta +1)^2}-\alpha +2 \gamma  \delta +\gamma +\delta +1}{2 \gamma  (\gamma +1)} \right )$ 

и

$\left ( \frac{\sqrt{\alpha ^2-2 \alpha  (2 \gamma  \delta +\gamma +\delta +1)+(\gamma +\delta +1)^2}-\alpha +\gamma -\delta +1}{2 (\gamma +1)}, \frac{\sqrt{\alpha ^2-2 \alpha  (2 \gamma  \delta +\gamma +\delta +1)+(\gamma +\delta +1)^2}-\alpha +2 \gamma \delta +\gamma +\delta +1}{2 \gamma  (\gamma +1)} \right )$ 

%%% Local Variables: 
%%% mode: latex
%%% TeX-PDF-mode: t
%%% TeX-master: "diploma"
%%% End: 
