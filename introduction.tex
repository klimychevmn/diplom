\chapters{Введение}

План: 

1. Найти положение равновесия исходной системы. Мы будем работать в окрестности положения равновесия $\left ( 0, \frac{\delta }{\gamma} \right )$. Выполнить сдвиг на это положение равновесия. Получить матрицу линейной части исходной системы. Найти условия, когда эта матрица имеет нулевое и отрицательное собственные значения.

2. В исходной системе перейти к собственному базису (чтобы матрица лин. части приняла диагональный вид). Разложить правую часть системы в ряд Тейлора в окрестности нуля до четвертых степеней включительно.

3. Найти собственные значения и собственные векторы матрицы линейной части исходной системы. Изобразить собственные подпространства и центральное многообразие на координатной плоскости. Объяснить, почему мы ищем центральное многообразие в виде степенного ряда, начиная со второй степени.

4. Применить метод центральных многообразий (с точностью до четвертых степеней включительно) и получить уравнение, эквивалентное исходной двумерной системе.

5. Найти положения полученного уравнения, производную его правой части и с помощью или второго метода Ляпунова исследовать их на устойчивость. Выписать условия существования и устойчивости найденных положений равновесия.

6. Изобразить бифуркационную диаграмму.

7. Построить графики решений исходной двумерной системы и эквивалентного ей уравнения, сравнить их.

%%% Local Variables: 
%%% mode: latex
%%% TeX-PDF-mode: t
%%% TeX-master: "diploma"
%%% End: 
